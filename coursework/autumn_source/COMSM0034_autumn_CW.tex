\documentclass[12pt]{article}
\usepackage{amsfonts, epsfig}
\usepackage[authoryear]{natbib}
\usepackage{graphicx}
\usepackage{fancyhdr}
\pagestyle{fancy}
\lfoot{\texttt{comsm0034.github.io}}
\lhead{IP\&B Coursework - Conor / Rui}
\rhead{\thepage}
\cfoot{}
\begin{document}

\section*{Information Processing and the Brain} 

For this course work you are asked to implement and explore the
behaviour of a classical supervised learning algorithm and discuss how
it relate to learning in the brain. You only need to select one of the
topics below​. You can choose your language of preference, but we would
recommend using Python or Matlab. There are two parts to the
coursework, the first requires you to implement backpropagation and
the second, more speculative part, asks you to test a simplification
of this algorithm which would make it more biologicaly plausible. Each
part is worth half the marks.

The backpropagation algorithm is often used in machine learning to
solve the credit assignment problem. Please implement a simple
feedforward network with one hidden layer and sigmoidal units to
analysis the widely used handwritten digit recognition dataset
(MNIST); you may use autograd to achieve this. You are welcome to use
code from stardard examples in your implementation; you should write a
paragraph describing your approach and a graph or table quantifying
the performance of the algorithm.

Straightforard attempts to argue that deep learning resembles neuronal
computation flounder because back-propagation requires that
information about the behavious of later layers is required to
determine learning in earlier layers. In this unit we have examined
some ways this problem can be avoided. However, a counter proposal
could hypothesis that learning does not need to occur in earlier
layers. According to this hypothesis the role of the earlier layers is
to provide a `dictionary' of recodings of the input which the final
layer can select from. The second part of this course work is to test
this idea: it seems unlikely to be true but if it was it would be easy
to suggest biological mechanisms to support this sort of learning.

One approach would be to train the network in the normal way and to
then shuffle the weights connecting the input layer to the middle
layer before retraining the network, this time only updating the
weights that go from the middle layer to the output layer. How does
this compare to the original? Who does this compare to a network
without a hidden layer?

Please email conor.houghton@bristol.ac.uk and rui.costa@bristol.ac.uk
with a pdf report of no more than one page.



\end{document}

